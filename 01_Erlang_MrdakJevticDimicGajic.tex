 % compile: pdflatex  -shell-escape Erlang_Jevtic_Mrdak_Dimic_Gajic.tex
% !TEX encoding = UTF-8 Unicode

\documentclass[a4paper]{article}

\usepackage{color}
\usepackage{url}
\usepackage[T2A]{fontenc} % enable Cyrillic fonts
\usepackage[utf8]{inputenc} % make weird characters work
\usepackage{graphicx}
\usepackage{amsthm}
\newtheorem{definition}{Definicija}

\usepackage[english,serbian]{babel}

\usepackage[unicode]{hyperref}
\hypersetup{colorlinks,citecolor=green,filecolor=green,linkcolor=blue,urlcolor=blue}

\newtheorem{primer}{Primer}[section]

\usepackage{minted}
\usemintedstyle{colorful}

\usepackage{tikz}
\def\checkmark{\tikz\fill[scale=0.4](0,.35) -- (.25,0) -- (1,.7) -- (.25,.15) -- cycle;}

\usepackage{listings}
\lstset{language=erlang}

\begin{document}

\title{Erlang - funkcionalno rešenje za konkurentni svet\\ \small{Seminarski rad u okviru kursa\\Metodologija stručnog i naučnog rada\\ Matematički fakultet}}

\author{Tijana Jevtić, Jelena Mrdak, David Dimić, Zorana Gajić\\
tijanatijanajevtic@gmail.com, mrdakj@gmail.com,\\daviddimic@hotmail.com, zokaaa\_gajich@bk.ru}
\date{6.~april 2019.}
\maketitle

\abstract{
Kroz ovaj rad čitalac će steći globalnu sliku o jeziku Erlang 
i detaljniji pogled na neke važne koncepte. 
Biće izložena potreba nastanka ovakog konkurentno-orijentisanog jezika u oblasti telekomunikacije. 
Kao takav je, ne samo dominantan u oblasti za koju je nastao, 
već je ušao i u širu upotrebu, pa i uticao na nastanak drugih popularnih jezika kao što je Eliksir. 
Erlangov koncept konkurentnosti kroz upotrebu funkcionalnog stila programiranja, 
rukovanje greškama
i lakoća rada sa procesima, portovima i distribuiranim programima čini ga jezikom vrednim pažnje i izučavanja.
}


\setcounter{tocdepth}{1} 
\tableofcontents

\newpage

\section{Uvod}
\label{sec:uvod} 
U oblasti telekomunikacije bilo je potrebno izgraditi dovoljno 
dobar programski jezik za pisanje brzih konkurentnih i distribuiranih programa i sistema \cite{book_joe}. 
Kao odgovor na ovakve zahteve nastao je i razvio se jezik Erlang \cite{phdthesis}.
Njegova svojstva i osobine koje ga izdvajaju od ostalih programskih jezika
biće prikazane u ovom radu iz nekoliko uglova, ilustrovane primerima \cite{book_fred, book_joe}.
Poseban akcenat biće na njegovoj konkurentnosti i obradi grešaka \cite{book_concurrent},
po čemu je karakterističan, a što je omogućila funkcionalna i deklarativna paradigma. Sa druge strane, biće pokazano kako se kroz okruženja može prilagoditi vebu \cite{ChicagoBossDocumentation, DjangoTempDoc, ZotonicDocumentation}.\\

% Opis po poglavljima 
Kroz niz poglavlja i primera, biće ispričana istorija Erlanga - kad, kako, gde i zašto je nastao \ref{sec:nastanak}, 
po čemu je karakterističan i koja mu je osnovna namena \ref{sec:namena}. 
Dalje, u poglavlju \ref{sec:osobine} biće opisane osnovne osobine i konstrukcije jezika, da bismo u poglavlju \ref{sec:konkurentnost} posvetili posebnu pažnju konkurentnosti i greškama u konkurentnim programima. 
U delu \ref{sec:okruzenja} upoznaćemo se sa okruženjima {\em ChicagoBoss}, {\em Nitrogen} i { \em Zotonic}. Potom ćemo videti kako se instalira i pokreće Erlang \ref{sec:instalacija}
i bolje upoznati sa jezikom kroz konkretne primere sa objašnjenjima \ref{sec:primeri}.


\section{Nastanak i istorijski razvoj}
\label{sec:nastanak}
Godine 1981. oformljena је nova laboratorija, Erikson CSLab (eng.~{\em The Ericsson CSLab}) u okviru firme Erikson sa
ciljem da predlaže i stvara nove arhitekture, koncepte i strukture za buduće softverske sisteme \cite{phdthesis}.
Eksperimentisanje sa dodavanjem konkurentnih procesa u programski jezik Prolog je bio jedan
od projekata Erikson CSLab-a i predstavlja začetak novog programskog jezika.
Taj programski jezik je 1987. godine nazvan Erlang
\footnote{ Erlang je jedinica saobraćaja u oblasti telekomunikacija 
i predstavlja kontinuirano korišćenje jednog kanala 
(na primer. ako jedna osoba obavi jedan poziv telefonom u trajanju od sat vremena, 
tada se kaže da sistem ima 1 Erlang saobraćaja na tom kanalu).}.    
Sve do 1990., Erlang se mogao posmatrati kao dijalekt Prologa. Od tada, Erlang
ima svoju sintaksu i postoji kao potpuno samostalan programski jezik.
Godine rada su rezultirale u sve bržim, boljim i stabilnijim verzijama jezika, kao
i u nastanku standardne biblioteke OTP (eng.~{\em The Open Telecom Platform}) \cite{phdthesis}.
Od decembra 1998. godine, Erlang i OTP su postali deo slobodnog softvera (eng.~{\em open source software})
i mogu se slobodno preuzeti sa Erlangovog zvaničnog sajta \cite{sajt}.
Danas, veliki broj kompanija koristi Erlang u razvoju
svojih softverskih rešenja. Neke od njih su: Erikson, Motorola, Votsap (eng.~{\em WhatsApp}), 
Jahu (eng.~{\em Yahoo!}), Fejsbuk (eng.~{\em Facebook}) \cite{sajt}.


\subsection{Uticaji}
\label{subsec:uticaji}
Erlang je funkcionalan, deklarativan i konkurentan programski jezik.
Na njega, kao na funkcionalan jezik, uticao je Lisp funkcionalnom paradigmom koju je 
prvi predstavio. Na planu konkurentnosti Erlang je svojevrstan primer (detaljnije u poglavlju \ref{sec:osobine}). \\ 

Na početku, Erlang je stvaran kao dodatak na Prolog, vremenom prerastao u 
dijalekt Prologa, a kasnije je zbog svoje kompleksnosti i sveobuhvatnosti evoluirao
u potpuno novi programski jezik. Stoga je uticaj Prologa na Erlang bio 
neminovan. Sintaksa Erlanga u velikoj meri podseća na Prologovu 
(na primer. promenljive moraju počinjati velikim slovom, 
svaka funkcionalna celina se završava tačkom), oba jezika u velikoj meri koriste poklapanje obrazaca
(eng.~{\em pattern matching}). \\

Sa druge strane, Erlang je uticao na nastanak programskog jezika Eliksir (eng.~{\em Elixir}). Eliksir,
uz izmenjenu Erlangovu sintaksu, dopunjenu Erlangovu standardnu biblioteku, uživa široku popularnost 
(\href{https://elixir-companies.com/}{lista kompanija koje ga svakodnevno koriste}). 

\section{Osnovna namena, svrha i mogućnosti}
\label{sec:namena}
Sa početkom od 1981. godine, jedan od zadataka Eriksonove laboratorije za računarstvo je bio pronalaženje načina za bolje programiranje aplikacija
za telekomunikacije \cite{phdthesis}. Takve aplikacije su ogromni programi i od velike važnosti je da rade sve vreme (koliko je to moguće). 
Naravno, poznato je da će tolika količina koda zasigurno imati greške, ali u toj vrsti industrije, greške mogu biti fatalne. Na primer, 
šta se dešava ako je došlo do kvara na nekoj telefonskoj liniji, a telefon nam je hitno potreban (recimo, neko ima srčani udar).
Jednostavno nije moguće zaustaviti takvu aplikaciju, popraviti je i nanovo pustiti u rad.
Kako se izboriti sa greškama u softverskim sistemima kada su one neminovne je osnovna motivacija za razvoj Erlanga \cite{phdthesis}. \\

Tako, jedna od njegovih namena jeste pisanje što sigurnijih programa koje je moguće popraviti bez potrebe za isključivanjem čitavog sistema \cite{book_joe}.
Vrlo brzi konkurentni i distribuirani programi su još jedna od Erlangovih specijalnosti. 
Poseban koncept konkurentnosti koji je implementiran u Erlangu (više u poglavlju \ref{sec:konkurentnost}), kao i funkcionalna paradigma 
omogućavaju lako skaliranje programa i pravljenje velikih konkurentnih i distribuiranih sistema.
Velika zajednica koja se godinama razvijala je doprinela stvaranju velikog broja biblioteka i okruženja za Erlang, te proširila njegov 
inicijalni skup mogućnosti i namena \cite{book_joe}. 


\section{Osnovne osobine}
\label{sec:osobine}
Sve paradigme koje podržava Erlang su tu da bi se dobio jednostavan, kvalitetan i
siguran konkurentan jezik.
Kao svaki funkcionalni jezik poseduje sakupljač otpadaka \cite{book_concurrent} u realnom vremenu, tako da se ne mogu pojaviti greške programera pri rukovanju memorijom. 
Sa konkurentne strane, sistem ima ugrađenu kontrolu vremena, 
u smislu da se može odrediti koliko će neki proces čekati na poruku pre nego što se aktivira, 
pa omogućava pisanje aplikacija koje rade u mekom realnom vremenu (eng.~{\em soft real-time systems}) \cite{book_concurrent} sa odzivom od nekoliko milisekundi. 
U ovom poglavlju videćemo koji su tipovi podržani u Erlangu da bi se njegove osobine i namene opisane u poglavlju \ref{sec:namena} ostvarile, 
kao i neka osnovna svojstva i koncepte.



\subsection{Tipovi i promenljive}
Erlang je dinamički i jako tipiziran jezik. 
Na raspolaganju nam je osam primitivnih tipova \cite{phdthesis}. 
Osim uobičajnih celobrojnih, realnih vrednosti i referenci, Erlang uvodi i neke specifične tipove:
\begin{list}{•}{}
\item Atomi koji se pišu malim slovima i predstavljaju konstante i enumerisane tipove. Samo ime je njihova vrednost. Pandan su makroima i enumerisanim tipovima u jeziku C.
\item Binarne vrednosti i bitovska sintaksa omogućavaju lako i čitljivo prelamanje broja na binarne segmente zadate širine. 
Na primer, boje obično označavamo heksadekadno u RGB oznaci kao tri bajta - jedan bajt za crvenu, jedan za plavu i jedan za zelenu boju.
Ako nam je data neka boja, recimo {\mintinline{erlang}{16#FF9A29}} i
želimo da vidimo koje vrednosti imaju pojedinačni delovi boje, možemo
prelomti broj na tri dela po osam bitova, odnosno na širinu 24, što se 
zapisuje kao {\mintinline{erlang}{<<16#FF9A29:24>>}}. Rezultat je {\mintinline{erlang}{<<255,154,41>>}}.
\item Identifikatori procesa predstavljaju reference na procese. Kreiraju se funkcijom spawn.
\item Portovi služe za komunikaciju sa spoljašnjim svetom. Ako su u skladu sa protokolom portova preko njih se mogu slati i primati poruke.
\end{list}
 
Tu su i dve osnovne strukture koje mogu da sadrže bilo koje tipove: torke
{\mintinline{erlang}{{Elem1, Elem2,... ElemN}}} za fiksirani broj elemenata u njima 
i liste {\mintinline{erlang}{[Elem1, Elem2,...]}} za čuvanje promenjivog broja elemenata. 
Osnovni operator konstrukcije liste je {\mintinline{erlang}{[Glava|Rep]}}. 
U okviru listi se prikazuju i niske, za koje ne postoji ugrađeni poseban tip, 
već su one liste vrednosti koje odgovaraju vrednostima karaktera. Ako svi elementi liste mogu da se prikažu kao karakteri onda će lista biti ispisana kao niska, što ilustruje naredni primer.
\begin{minted}{erlang}
1> [16#5A, 97+3, 2*50+14, 97, 8#166, 2#1101111].
"Zdravo"
2> [65,97,2].
[65,97,2]
\end{minted}
U drugom primeru se ne može prikazati kao karakter pa lista nije prikazana kao niska. 
Ovde vidimo i neka elementarna izračunavanja i kako sa \# možemo elegantno koristiti bilo koju brojevnu osnovu. 
Da bismo sačuvali izračunavanja potrebne su nam promenljive.\\

Promenljive mogu biti vezane (eng.~{\em bound}), one kojima je ,,dodeljena'' neka vrednost, i slobodne. 
Vezivanje se vrši najviše jednom i vrednost vezanih promenljivih više se ne može menjati (eng.~{\em single assignment variables}) 
osim ako se u interpreteru ne pozove funkcija  {\mintinline{erlang}{f()}} koja sve promenljive načini slobodnim \cite{book_fred}. 
Ovo je u skladu sa idejom funkcionalnih jezika da nema sporednih efekata 
što za posledicu ima jednostavno izvođenje konkurentnosti, 
iako Erlang nije čisto funkcionalan jezik.
Zapravo, operator '=' ne predstavlja nikakvu dodelu već poklapanje obrazaca.


\subsection{Poklapanje obrazaca}
Većinu funkcija u Erlangu, kao i svako vezivanje promenljivih pišemo putem poklapanja obrazaca (eng.~{\em pattern matching}). Neformalno\footnote{ Formalna definicija u dodatku \ref{sec:poklapanje_obrazaca_formalno}}, to je postupak poređenja terma sa obrascem. Ako obrazac i term imaju isti oblik, poklapanje uspeva, pri čemu će svaka promenljiva biti vezana sa podatkom na njemu odgovarajućoj poziciji. 
Ovaj proces poznat je kao unifikacija. 
Pri unifikaciji na raspolaganju je i posebna anonimna promenljiva '\_' koju koristimo kada nas neka vrednost ne zanima i ne želimo nijednu promenljivu da vežemo za tu vrednost. U sledećem primeru prve tri linije pokazuju uspešnu unifikaciju, dok je u poslednjoj pokušana unifikacija {\mintinline{erlang}{X}} sa {\mintinline{erlang}{51}}, što nije uspelo kako je {\mintinline{erlang}{X}} već vezano za {\mintinline{erlang}{{137, 42}}}.

\begin{minted}{erlang}
1> Z = 2.
2> {X, macka} = {{137, 42}, macka}.
3> [Glava|_] = [1,2,3,4,5,6].
4> {X, Y} = {51, kuce}.
\end{minted}

Jedno proširenje mogućnosti poklapanja obrazaca, korisno za izvođenje jednostavnih testova i poređenja u šablonu daju nam čuvari (eng.~{\em guards}) sa ključnom rečju {\mintinline{erlang}{when}}. Čuvari su izrazi odvojeni sa ',' koji sadrže samo predikate poređenja\footnote{ Operatori poređenja su: <, =<, >, =>, ==, /=, =:=, =/=} kao u sledećem primeru.
\begin{minted}{erlang}
max(X, Y) when X > Y -> X;
max(X, Y) -> Y.
\end{minted}
Funkcija {\mintinline{erlang}{max(X,Y)}} je definisana preko poklapanja obrazaca uz korišćenje čuvara jednog uslova koji određuje povratnu vrednost. U narednom delu formalizovaćemo šta su funkcije u Erlangu.


\subsection{Funkcije}
Jedna definicija funkcije može se podeliti u više slučajeva
razdvojenih sa ';' do kojih se dolazi putem poklapanja obrazaca.
Obrasci se mogu naći u argumentima funkcije ili u opcionom čuvaru,
osnosno {\mintinline{erlang}{when}} delu. 
Telo funkcije sastoji se od niza izraza razdvojenih sa ','.
Tačkom se završava definicija i odvaja od ostalih funkcija. 
Njihovi primeri i korišćenja biće detaljnije opisani u delu \ref{sec:primeri}.
\begin{minted}{erlang}
ime_funckije(a11, a12, ... a1N) [when g11, g12, ... g1N] -> 
	telo1;
...
ime_funckije(aM1, aM2, ... aMN) [when gM1, gM2, ... gMN] -> 
	teloM.
\end{minted}

Kao u svakom funkcionalnom jeziku funkcije su građani prvog reda, 
tako da ih možemo prosleđivati kao argumente, vraćati kao povratnu vrednost, itd. Na raspolaganju su nam i anonimne funkcije koje imaju sledeći oblik:
\begin{minted}{erlang}
fun(a1, a2, ... aN) -> telo end.
\end{minted}

Česte su rekurzivne definicije funkcija \cite{book_joe}, 
ali moramo imati na umu da su najefikasnije repne rekurzivne za koje nije potreban stek. 
Mnoge funkcije u Erlangu dizajnirane su da se izvršavaju u beskonačnim petljama, 
posebno u klijent-server modelu u ulozi servera opisanog u narednom poglavlju \ref{sec:konkurentnost}.


\section{Konkurentnost kao glavna specifičnost}
\label{sec:konkurentnost}
Jedna od osnovnih osobina Erlanga i specifičnost po kojom se izdvaja od drugih jezika je konkurentnost i koncept na kom je zasnovana.
Posmatrajući svet oko
sebe, uviđamo da je on u suštini konkurentan - u istom trenutku se dešava veliki broj procesa \cite{phdthesis}. 
U tom istom trenutku, mi smo sposobni da takav svet pojmimo i odreagujemo na sve što se u njemu dešava. Dakle, mi prirodno razumemo konkurentnost.
Tako se i prirodno nameće potreba za programskim jezikom koji bi omogućavao jednostavno modelovanje sveta kakav on stvarno jeste.

\subsection{Koncept}
Koncept konkurentnosti implementiran u Erlangu se zove konkurentnost slanjem poruka (eng.~{\em message passing concurrency}), šematski prikazan na slici \ref{fig:concurrency}. Ovo podrazumeva postojanje velikog broja procesa koji nikad ne dele memoriju, već komuniciraju isključivo asinhronim slanjem poruka \cite{book_joe}.
Sva izračunavanja se obavljaju u okviru procesa i trebalo bi da sistem bude dizajniran tako da jedan proces radi jedan mali posao.
Važno je napomenuti da procesi u Erlangu nisu procesi operativnog sistema, već Erlanga.
To je moguće zbog toga što se njegovi programi izvršavaju na BEAM virtuelnoj mašini.
Procesi se prave i uništavaju jako brzo, zauzimaju samo onoliko memorije koliko je neophodno (u većini slučajeva vrlo malo)
i ponašaju se isto na svim operativnim sistemima \cite{book_joe}.

\begin{figure}[h!]
\begin{center}
\includegraphics[scale=0.5]{concurrency.png}
\end{center}
\caption{Konkurentnost u Erlangu}
\label{fig:concurrency}
\end{figure} 

\subsection{Slanje i primanje poruka}
%Slanje
Erlang omogućava jednostavno kreiranje novog procesa pozivom funkcije {\mintinline{erlang}{spawn(...)}} koja vraća pid (eng.~{\em process identifier}) na osnovu kojeg svaki proces može da razlikuje ostale procese.
\begin{minted}{erlang}
Pid = spawn([Module], FunctionName, [ArgumentList]) 
\end{minted}

Jedini način za ostvarivanje komunikacije između dva procesa je putem slanja poruka korišćenjem operatora '!'. U {\mintinline{erlang}{Pid ! Message}} procesu sa identifikatorom {\mintinline{erlang}{Pid}} šalje se poruka sadržana u promenljivoj {\mintinline{erlang}{Message}}, 
pri čemu poruka može biti bilo koji validni term. 
Operator slanja evaluira svoje argumente – prvo levi argument da bi se dobio pid procesa koji prima poruku, 
a potom desni da bi se dobila poruka koja se šalje. 
Povratna vrednost je baš ta poruka.
Slanje poruka je asinhrono, pošiljalac neće čekati da poslata poruka stigne na odredište niti da bude primljena \cite{book_joe, book_concurrent}.\\

%Primanje
Za primanje poruka koristi se operator {\mintinline{erlang}{receive}}. 
Svaki proces ima svoje sanduče gde se nalaze poruke u redosledu pristizanja. 
Poruke se porede poklapanjem obrazaca. 
Kada poklapanje uspe poruka se vadi iz sandučeta i izvršava se navedena akcija. 
{\mintinline{erlang}{Receive}} vraća vrednost poslednjeg izraza iz izvršene akcije.
Ako poslata poruka nikad ne stigne na odredište, 
što se može desiti na primer kao posledica pada procesa koji je poslao poruku, 
izbegavamo beskonačno čekanje tako što se uvodi maksimalno vreme čekanja (eng.~{\em timeout}).
\begin{minted}{erlang}
receive
    Pattern1 [when Guard1] ->
        Expressions1;
    Pattern2 [when Guard2] ->
        Expressions2;
	... 
	[after Time -> 
		Expressions]
end
\end{minted}

Ukoliko želimo da primimo bilo koju poruku to može da se uradi korišćenjem {\mintinline{erlang}{AnyMessage}}, 
ali češće želimo da primamo samo one poruke koje su namenjene nama. 
Da bismo to ostvarili moramo poslati svoj pid 
(ako šaljemo pismo moramo napisati svoju adresu da bismo dobili odgovor) 
što se postiže sa funkcijom {\mintinline{erlang}{self()}}.
\begin{minted}{erlang}
Pid ! {self(), Message}
\end{minted}

\subsection{Klijent-server model}
\label{sec:ksmodel}
Kada šaljemo poruku procesu moramo znati njegov pid. 
Ovo često nije praktično zbog mogućeg velikog broja procesa, 
niti poželjno iz bezbednosnih razloga 
(neki procesi bi trebalo da sakriju svoj identitet). 
Da bi bilo omogućeno slanje poruka bez poznavanja identifikatora uvodi se pojam registrovanja procesa, 
tj. davanje imena koje mora biti atom \cite{book_joe, phdthesis}.
Funkcijom {\mintinline{erlang}{register(name, Pid)}} atom {\mintinline{erlang}{name}} se vezuje za {\mintinline{erlang}{Pid}} i na dalje ga možemo identifikovati preko dodeljenog atoma.\\

Glavni razlog registrovanja procesa je da se omogući klijent-server model koji je ključni za komunikaciju između procesa u Erlangu \cite{book_concurrent, book_joe, phdthesis}.
U ovom modelu obe strane mogu biti procesi na istoj ili različitim mašinama. 
Klijent uvek započinje neko izračunavanje obraćajući se serveru, 
koji obrađuje zahtev i vraća odgovor klijentu. Jedan jednostavan primer ovog modela biće objašnjen u delu \ref{sec:primeri}.

\subsection{Greške u konkurentnim programima}
Za pravljenje sistema otpornog na greške neophodna su nam dva računara, 
jedan {\em radnik} koji će izvršavati posao i drugi {\em supervizor} 
koji će posmatrati i biti spreman da preuzme posao u trenutku kada se dogodi pad prvog  (eng.~{\em worker-supervisor relationship}) \cite{book_joe}. 
U Erlangu je ovo omogućeno korišćenjem povezivanja procesa funkcijom {\mintinline{erlang}{link(...)}}.
Pretpostavićemo da su neki procesi A i B  povezani. 
Ako proces A iznenada prestane sa radom onda se proces B obaveštava takozvanim završnim signalom. 
Ukoliko je obrnuto, obaveštava se proces A. 
Šta se dešava kada proces dobije završni signal? 
Ako proces nije bio aktivan u tom trenutku onda se on uništava, 
a inače proces može da traži da se uhvate signali. Proces u ovakvom stanju se naziva sistemskim procesom \cite{book_joe}. \\

Sledeći primer prikazuje kako uhvatiti završni signal. 
Link ka procesu {\mintinline{erlang}{Pid}} pravi se pomoću funkcije
{\mintinline{erlang}{on_exit(Pid, Fun)}}. 
Pozivom funkcije {\mintinline{erlang}{process_flag(...)}} novokreirani proces se transformiše u sistemski proces 
i onda se povezuje sa procesom {\mintinline{erlang}{Pid}} uz pomoć funkcije {\mintinline{erlang}{link(Pid)}}. 
Kada proces umre onda je primljen i obrađen blokom {\mintinline{erlang}{receive}}.
\begin{minted}{erlang} 
on_exit(Pid, Fun) ->	
    spawn(fun() ->    
                   process_flag(trap_exit, true),
                   link(Pid),
                   receive
                       {'EXIT', Pid, Why} ->
                              Fun(Why)
                   end
          end).
\end{minted}

Testiranje se vrši kreiranjem funkcije {\mintinline{erlang}{F}}
koja čeka na poruku {\mintinline{erlang}{X}} od koje onda kreira atom i nekog procesa {\mintinline{erlang}{Pid}}. 
Potom se za nadgledanje postavlja funkcija {\mintinline{erlang}{on_exit(...)}.
\begin{minted}{erlang} 
1> F = fun() ->
     receive
         X -> list_to_atom(X)
     end
   end.

2> Pid = spawn(F).

3> on_exit(Pid,
           fun(Why) ->
               io:format(" ~p died with:~p~n",[Pid, Why])
           end).

\end{minted}

Ako se procesu {\mintinline{erlang}{Pid}} pošalje atom, on će umreti jer će pokušati da izračuna funkciju {\mintinline{erlang}{list_to_atom(...)}}
sa argumentom koji nije lista i onda će funkcija {\mintinline{erlang}{on_exit(...)}} prijaviti grešku.
\begin{minted}{erlang} 
4> Pid ! Zdravo.
Zdravo
<0.61.0> died with:{badarg,[{erlang,list_to_atom,[hello]}]}
\end{minted}

Još jedan način za razrešavanje grešaka je takozvanim živo-održivim procesima (eng.~{\em a keep-alive process}). 
Ideja je kreirati registrovane procese, koje smo spomenuli u \ref{sec:ksmodel}, 
koji će uvek biti živi i ako prestanu sa radom iz bilo kog razloga, 
odmah će se restartovati. U narednom primeru se kreira registrovani proces pod nazivom {\mintinline{erlang}{Name}} koji izvršava {\mintinline{erlang}{spawn(Fun)}}. 
\begin{minted}{erlang} 
keep_alive(Name, Fun) ->
    register(Name, Pid = spawn(Fun)),
    on_exit(Pid, fun(_Why) -> keep_alive(Name, Fun) end).
\end{minted}
 

\section{Okruženja i njihove karakteristike}
\label{sec:okruzenja}
Erlang je poznat kao jezik koji podržava skalabilne sisteme koji su otporni na greške (eng.~{\em scalable fault-tolerant systems}), 
ali takođe nudi mnoštvo mogućnosti koje ga čine dobrim jezikom za veb programiranje. Na primer, mogućnost reagovanja na više korisničkih zahteva istovremeno, 
ne razmišljajući o problemima konkurentnosti.
U tabeli \ref{tab:tabela_okruzenja} je prikazano poređenje tri glavna veb okruženja: {\em ChicagoBoss}, {\em Nitrogen} i {\em Zotonic} po nekim interesantnim osobinama.

\begin{table}[h!]
\begin{center}
\caption{Poređenje Erlang veb okruženja}
\begin{tabular}{|c|c c c|}\hline
 &ChicagoBoss &Nitrogen &Zotonic \\ \hline
Razvoj zasnovan na događajima &\checkmark  &\checkmark & \checkmark  \\ 
Okruženje za testove &\checkmark  &\checkmark & \checkmark  \\ 
Generisanje koda &\checkmark & - & - \\ 
Đango šabloni &\checkmark & - &\checkmark  \\
Integrisani mejl server &\checkmark & - &\checkmark  \\ 
UTF-8 u Erlang kodu &\checkmark & - &\checkmark  \\ 
Višejezični podaci & - & - &\checkmark  \\ 
Generisanje {\em JavaScript} koda & - &\checkmark & \checkmark  \\ 
Generisanje {\em JSON} formata &\checkmark  &\checkmark & \checkmark  \\ 
Integrisani {\em WebSocket} &\checkmark  &\checkmark & \checkmark  \\ \hline
 \end{tabular}
\label{tab:tabela_okruzenja}
\end{center}
\end{table} 

Okruženje {\em ChicagoBoss} sadrži sloj apstrakcije baze podataka (eng.~{\em database abstraction layer}) pod nazivom {\em BossDB} \cite{ChicagoBossDocumentation} koji je zaslužan za postavljanje upita nad bazom podataka i njeno ažuriranje. 
Podržani su {\em MySQL, Mnesia, Tokyo Tyrant i PostgreSQL}. 
Za razliku od {\em ChicagoBoss-a}, {\em Nitrogen} okruženje ne podržava model podataka uopšte, dok {\em Zotonic} \cite{ZotonicDocumentation} podržava isključivo {\em PostgreSQL}.\\

Takođe, interesantno je primetiti da neka okruženja imaju integrisani mejl server koji nudi funkcije za primanje i slanje e-pošte i ostale mogućnosti, 
čime olakšava rad korisnicima. 
Na primer, slanje e-pošte u okruženju {\em ChicahoBoss} izgleda ovako:

\begin{minted}{erlang}
boss_mail:send(FromAddress, ToAddress, Subject, Body)
\end{minted} 

U tabeli \ref{tab:tabela_okruzenja} vidimo da sva tri okruženja podržavaju 
i okruženje za testove, gde su testovi strukturirani kao stabla nadovezivanja (eng.~{\em trees of continuations}) \cite{EvanMillerTesting}. 
Postoje već implementirane funkcije koje olakšavaju testiranje nekih opšte poznatih akcija kao što je 
provera da li je e-pošta ispravno primljena/poslata, da li je stranica na vebu modifikovana,  itd. \\

Đango šabloni (eng.~{\em Django templates}) \cite{DjangoTempDoc} služe za jednostavnije i brže generisanje dinamičkih HTML stranica pomoću gotovih šablona.
 {\em Nitrogen} ima svoje {\em Nitrogen HTML} šablone ali je u procesu prelazak na Đango šablone.\\
 
Svaki od opisanih okruženja ima svoje prednosti i mane, 
te zato nije jednostavno presuditi koji od ovih okruženja treba koristiti zasigurno, a koji ne treba. 
U zavisnosti od onoga šta je prioritet bira se odgovarajuće okruženje. 
U dodatku \ref{sec:dodatak_primeri_okruzenja} možete pogledati primere u nekim od spomenutih okruženja.

\section{Instalacija i pokretanje}
\label{sec:instalacija}
Postoji više načina da se instalira Erlang sa neophodnim paketima.
Ovde će biti predstavljena instalacija korišćenjem prekompajliranih binarnih fajlova 
za neke operativne sisteme zasnovane na Linuksovom jezgru i pokretanje na jednom od njih, kao i instalacija za Vindouz (eng.~{\em Windows}).


\subsection{Linuks}
\label{subsec:instalacijaLinux}
Na operativnim sistemima zasnovanim na {\em Ubuntu}, Erlang se može instalirati sa:
{\mintinline{bash}{sudo apt-get install erlang}}. 
Nakon uspešne instalacije, Erlang k\^ od je moguće kompajlovati
ili interpretirati i pokretati u interpretatoru.
Interpretator se pokreće kucanjem komande {\mintinline{bash}{erl}} u terminalu, a iz istog
se izlazi sa {\mintinline{bash}{Ctrl+G}} iza kog sledi {\mintinline{bash}{q}} \cite{book_joe}.
Erlang interpretator ima u sebi ugrađen editor teksta koji je baziran na Emaksu (eng.~{\em Emacs}) \cite{book_fred}.
K\^od iz datoteke se kompajluje komandom {\mintinline{erlang}{erlc}} i navođenjem imena fajla sa ekstenzijom {\em .erl}.
Nakon toga se dobija izvršna datoteka sa ekstenzijom {\em beam} koja se može
pokrenuti uz navođenje adekvatnih argumenata komandne linije (eng.~{\em flags}).

\subsection{Vindouz}
\label{subsec:instalacijaWindows}
Na operativnom sistemu Vindouz, Erlang se može instalirati preuzimanjem binarne datoteke sa zvaničnog sajta \cite{sajt} programskog jezika. Posle duplog klika na {\em .exe} fajl samo je potrebno ispratiti uputstva. Pokretanje interpretatora se vrši na isti način kao i na Linuks sistemima.


\section{Primeri kodova sa objašnjenjima}
\label{sec:primeri}
Počećemo od primera {\em ,,Hello World''} i videti osnovnu sintaksu jezika. 
Jednolinijski komentari počinju znakom \%. 
Prvo navodimo naziv našeg modula u kome se nalaze funkcije koje pišemo, 
a da bi one mogle da se koriste izvan modula potrebno je da ih navedemo u {\mintinline{erlang}{-export}} naredbi. {\mintinline{erlang}{start/0}} označava da funkcija {\mintinline{erlang}{start()} prima nula argumenata.
Da bismo željeni tekst prikazali u konzoli, koristimo {\mintinline{erlang}{io}} modul koji sadrži potrebne IO funkcije u Erlangu.
\begin{minted}{erlang}
% hello world program
-module(helloworld). 
-export([start/0]). 

start() -> 
   io:fwrite("Hello, world!\n").

\end{minted}

Omogućene su sve klasične funkcionalne konstrukcije jezika. 
Pogledajmo samo neke od njih, kao što je razumevanje listi (eng.~{\em list comprehension}), funkcija {\mintinline{erlang}{map(F,L)}} i implementacija neke funkcije kao što je {\em QuickSort}.\\

Izdvajanje svih celobrojnih vrednosti iz liste većih od nekog broja putem razumevanja liste sastoji se od generatora {\mintinline{erlang}{<-}} i filtera liste.
\begin{minted}{erlang}
> [X || X <- [{1,5}, 2, 3, 4, crveno, 5, 6], integer(X), X>3].
[4,5,6]
\end{minted}

Primetimo da liste mogu da sadrže elemente različitog tipa.
Pogledajmo funcionalni način obrade liste putem funkcije map - njenu implementaciju i primenu.
{\mintinline{erlang}{map(F, List)}} vraća novu listu dobijenu primenom funkcije {\mintinline{erlang}{F}} na svaki element liste {\mintinline{erlang}{L}}.
\begin{minted}{erlang}
map(F, [H|T]) -> [F(H)|map(F, T)];
map(F, [])    -> [].

double(L)  -> map(fun(X) -> 2*X end, L).

> double([1,2,3,4]).
[2,4,6,8]
\end{minted}

Sortirajmo listu putem {\em QuickSort} algoritma. 
Koristeći poklapanje obrazaca razdvajamo definiciju na slučaj kada imamo bar jedan 
element i slučaj prazne liste. Kada imamo jedan element možemo ga proglasiti pivotom i u odnosu na njega podeliti na sve manje i sve veće od njega razumevanjem liste.
Potom rekurzivno obrađujemo dalje podliste.
\begin{minted}{erlang}
sort([Pivot|T]) ->
    sort([ X || X <- T, X < Pivot]) ++
    [Pivot] ++
    sort([ X || X <- T, X >= Pivot]);
sort([]) -> [].
\end{minted}

%primer konkurentnost
Pošto smo se upoznali sa elementarnim konstrukcijama, pogledajmo nešto što je 
karakteristično za Erlang. Naredni primer pokazuje kako komuniciraju procesi u klijent-server modelu opisanog u delu \ref{sec:konkurentnost}.
Klijent šalje zahtev serveru za računanje hipotenuze ili površine pravouglog trougla,
a server vrši izračunavanje u funkciji {\mintinline{erlang}{loop()}} i vraća odgovoru klijentu. 
Kada modul {\mintinline{erlang}{trougao}} bude implementiran, izvršavanje će izgledati ovako: U liniji 1 kreiramo novi serverski proces i dobijemo njegov pid, 
potom ga registrujemo i pozovemo funkciju {\mintinline{erlang}{klijent(...)}} koja enkapsulira slanje zahteva i primanje odgovora.
\begin{minted}{erlang}
1> Pid = spawn(fun trougao:loop/0).
2> register(server, Pid).
3> trougao:klijent(server, {hipotenuza,3,4}).
5.0
\end{minted}

U klijentskoj funkciji pošiljalac mora da uključi svoju adresu sa {\mintinline{erlang}{self()}} i pošalje zahtev na serverski proces.
Potom čeka odgovor koji je namenjen njemu, odnosno prihvata odgovor kada se poklopi obrazac torke {\mintinline{erlang}{{Pid, Response}}}. Tačnije, {\mintinline{erlang}{Pid}} je vezan, a {\mintinline{erlang}{Response}} je slobodna promenljiva koja će biti vezana kada stigne odgovor.
\begin{minted}{erlang}
klijent(Pid, Request) ->
    Pid ! {self(), Request},
    receive
	{Pid, Response} ->
	    Response
    end.
\end{minted}

Sa klijentske strane takođe se vrši poklapanje obrazaca sa atomom koji je poslat u zahtevu (šta klijent želi da računa), 
izračunavanje i slanje odgovora na adresu klijenta.
Kompletan kod dostupan je u datoteci \href{https://raw.githubusercontent.com/mrdakj/msnr/master/trougao.erl}{trougao.erl}
\begin{minted}{erlang}
loop() ->
    receive
	{From, {hipotenuza, A, B}} -> 
	    From ! {self(), sqrt(A*A + B*B)}, loop();
	{From, {povrsina, A, B}} -> 
	    From !  {self(), A * B / 2}, loop();
	{From, Other} ->
	    From ! {self(), {error, Other}}, loop()
    end.
\end{minted}

\section{Zaključak}
\label{sec:zakljucak}
U ovom radu su ukratko, ali jezgrovito objašnjene najvažnije osobine i paradigme programskog jezika Erlang: 
od upoznavanja sa sintaksom, preko elementarnih primera, do njegovog specifičnog koncepta konkurentnosti. 
Elegantan i jednostavan rad sa procesima u potpunosti je opravdao ideju i namenu za koju je jezik nastao.
Ipak, nijedna tehnologija, niti programski jezik nisu univerzalno rešenje, 
pa tako nije ni Erlang. Njegova ekspertiza su sigurni konkurentni i distribuirani
sistemi, aplikacije za telekomunikaciju, 
Internet serveri, dok nije najbolji izbor za obradu slika, signala i velike količine podataka \cite{sajt}. 
Kroz izložene koncepte ovaj rad pruža odličan uvod za upoznavanje sa Erlangom 
i podstrek čitaocu na dalje i dublje istraživanje koje bi se moglo ticati distribuiranosti, radu sa portovima, obradom grešaka i slično.


\addcontentsline{toc}{section}{Literatura}
\appendix
\bibliography{seminarski} 
\bibliographystyle{plain}

\newpage
\appendix
\section{Dodatak}

\subsection{Poklapanje obrazaca}
\label{sec:poklapanje_obrazaca_formalno}
Da bismo objasnili ovaj ključni koncept potrebno je prvo da definišemo pojmove terma i obrasca \cite{phdthesis}.

\theoremstyle{definition}
\begin{definition}{Osnovni term (eng.~{\em ground term})}
se definiše kao primitivni tip, uređeni par ili lista osnovnog terma.
\end{definition}

\begin{definition}{Obrazac ili šablon (eng.~{\em pattern})}
može biti primitivni tip, promenljiva, uređeni par ili lista šablona.
Ako su u obrascu sve promenljive različite onda se on naziva primitivnim.
\end{definition}

\begin{definition}{}
Ako je P  primitivni obrazac i T term, onda kažemo da se obrazac P poklapa sa T ako i samo ako:
\begin{list}{•}{}
\item Ako je P lista sa glavom Pg i repom Pr, a T lista sa glavom Tg i repom Tr, onda Pg mora da se unifikuje sa Tg i Pr sa Tr
\item Ako je P torka {P1,P2,...Pn} i T torka {T1,T2,...Tn}, onda svi elementi redom mora da se unifikuju – P1 sa T1, P2 sa T2, ... Pn sa Tn.
\item Ako je P konstanta, onda T mora da bude ista konstanta
\item Ako je P slobodna promenljiva V onda će V biti vezana za T
\end{list}
\end{definition}



\subsection{Primeri koda u veb okruženjima}
\label{sec:dodatak_primeri_okruzenja}
U poglavlju \ref{sec:okruzenja} smo se upoznali sa glavnim predstavnicima Erlang veb okruženja i videli njihove osnovne karakteristike. 
Prikazaćemo nekoliko primera u veb okruženjima. \\ 

Prvo ćemo pokazati jedan napredniji primer programa {\em ,,Hello World''} u {\em ChicagoBoss} okruženju. Erlang fajl bi izgledao ovako:
\begin{minted}{erlang}
-module(appname_my_controller, [Req]).
-compile(export_all).
hello('GET', [Name]) ->
	{ok, [{name, Name}]}.
\end{minted}

Dok je u HTML fajlu neophodno napisati:
\begin{minted}{html}
Hello, {{ name }}.
\end{minted} 

Primer u okruženju {\em Nitrogen} sa ispisom poruke kao reagovanje na klik. 
Prvo se učitava gotov šablon HTML strane i postavlja se naslov strane. 
Zatim se dodaje tekst ispod kojeg se kreira dugme sa njegovim identifikatorom i obezbeđuje se reagovanje na klik. 
Postoji hvatač događaja {\mintinline{erlang}{event(click)}} koji će ispisati poruku kada se klikne na dugme 
i zameniti sve ono što se nalazi u telu (eng.~{\em html body}) sa novom porukom.
\begin{minted}{erlang}
-module(module_name).
-compile(export_all).
-include_lib("nitrogen_core/include/wf.hrl").

main() -> #template{file=template_name}.
title() -> "My Hello World".

body() ->
    #panel{body=[
        "Click on button below!!!"
        #br{},
        #button{
            id=button_id_name,
            postback=click,
            text="Click Me"
        }
    ]}.

event(click) ->
    NewBody = #panel{
        id=id_name_of_the_replacement,
        text="You clicked!!!"
    },
    wf:replace(button_id_name, NewBody).
\end{minted}


\end{document}
