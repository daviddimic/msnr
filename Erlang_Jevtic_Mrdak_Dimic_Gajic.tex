% !TEX encoding = UTF-8 Unicode

\documentclass[a4paper]{article}

\usepackage{color}
\usepackage{url}
\usepackage[T2A]{fontenc} % enable Cyrillic fonts
\usepackage[utf8]{inputenc} % make weird characters work
\usepackage{graphicx}

\usepackage[english,serbian]{babel}

\usepackage[unicode]{hyperref}
\hypersetup{colorlinks,citecolor=green,filecolor=green,linkcolor=blue,urlcolor=blue}

\newtheorem{primer}{Primer}[section]

\usepackage{listings}
\lstset{language=erlang}

\begin{document}

\title{Erlang - funkcionalno rešenje za konkurentni svet\\ \small{Seminarski rad u okviru kursa\\Metodologija stručnog i naučnog rada\\ Matematički fakultet}}

\author{Tijana Jevtić, Jelena Mrdak, David Dimić, Zorana Gajić\\
tijanatijanajevtic@gmail.com, mrdakj@gmail.com,\\daviddimic@hotmail.com, zokaaa\_gajich@bk.ru}
\date{6.~april 2019.}
\maketitle

\abstract{
U ovom radu je prikazan programski jezik Erlang iz različitih uglova.
Kroz niz poglavlja i primera, ispričana je njegova istorija - kad, kako, gde i zašto je nastao, 
po čemu je karakterističan, šta ga to izdvaja od drugih programskih jezika, koji su to 
koncepti koji su svojevrsni Erlangu. \\
Nakon čitanja rada, čitalac će imati globalnu sliku o jeziku i detaljniji pogled na neke važne koncepte, 
kao i uvid u korišćenu literaturu koju može konsultovati radi daljeg informisanja o temi.

\setcounter{tocdepth}{1} 
\tableofcontents

\newpage

\section{Uvod}
\label{sec:uvod}

\begin{primer} I tabele treba da budu u svom okruženju, i na njih je neophodno referisati se u tekstu. Na primer, u tabeli \ref{tab:tabela1} su prikazana različita poravnanja u tabelama.

\begin{table}[h!]
\begin{center}
\caption{Razlčita poravnanja u okviru iste tabele ne treba koristiti jer su nepregledna.}
\begin{tabular}{|c|l|r|} \hline
centralno poravnanje& levo poravnanje& desno poravnanje\\ \hline
a &b&c\\ \hline
d &e&f\\ \hline
\end{tabular}
\label{tab:tabela1}
\end{center}
\end{table}

\end{primer}

\section{Nastanak i istorijski razvoj}
\label{sec:nastanak}

1981. godine je oformljena nova laboratorija, Erikson CSLab (eng.~{\em The Ericsson CSLab}) u okviru firme Erikson sa
ciljem da predlaže i stvara nove arhitekture, koncepte i strukture za buduće softverske sisteme.
Eksperimentisanje sa dodavanjem konkurentnih procesa u programski jezik Prolog je bio jedan
od projekata Erikson CSLab-a i predstavlja začetak novog programskog jezika.
Taj programski jezik je 1987. godine nazvan Erlang
\footnote{Erlang je jedinica saobraćaja u oblasti telekomunikacija 
i predstavlja kontinuirano korišćenje jednog kanala 
(npr. ako jedna osoba obavi jedan poziv telefonom u trajanju od sat vremena, 
tada se kaže da sistem ima 1 Erlang saobraćaja na tom kanalu).}.    
Sve do 1990., Erlang se mogao posmatrati kao dijalekt Prologa. Od tada, Erlang
ima svoju sintaksu i postoji kao potpuno samostalan programski jezik.
Godine rada su rezultirale u sve bržim, boljim i stabilnijim verzijama jezika, kao
i u nastanku standardne biblioteke OTP (eng.~{\em The Open Telecom Platform}) \cite{phdthesis}.
Od decembra 1998. godine, Erlang i OTP su postali deo slobodnog softvera (eng.~{\em open source software})
i mogu se slobodno preuzeti sa Erlangovog zvaničnog sajta \cite{sajt}.
Danas, veliki broj kompanija koristi Erlang u razvoju
svojih softverskih rešenja. Neke od njih su: Erikson, Motorola, Votsap (eng.~{\em Whatsapp}), 
Jahu (eng.~{\em Yahoo!}), Amazon, Fejsbuk (eng.~{\em Facebook}).


\subsection{Uticaji drugih programskih jezika}
\label{subsec:uticaji}

Erlang je funkcionalan i konkurentan programski jezik.
Na njega, kao na funkcionalan jezik, uticao je Lisp funkcionalnom paradigmom koju je 
prvi predstavio. Na planu konkurentnosti Erlang svojevrstan primer (detaljnije u odeljku \ref{sec:osobine}). \\
Na početku, Erlang je stvaran kao neki dodatak na Prolog, vremenom prerastao u 
dijalekt Prologa, a kasnije je zbog svoje kompleksnosti i sveobuhvatnosti evoluirao
u potpuno novi programski jezik. Stoga je uticaj Prologa na Erlang bio 
neminovan. Sintaksa Erlanga u velikoj meri podseća na Prologovu 
(npr. promenljive moraju počinjati velikim slovom u oba jezika, 
svaka funkcionalna celina se završava tačkom), oba jezika u velikoj meri koriste poklapanje obrazaca
(eng.~{\em pattern matching}). \\
Sa druge strane, Erlang je uticao na nastanak programskog jezika Eliksir (eng.~{\em Elixir}).

\section{Osnovna namena, svrha i mogućnosti}
\label{sec:namena}


\section{Osnovne osobine}
\label{sec:osobine}
 

\subsection{Podržane paradigme}
\label{subsec:paradigme}


\subsection{Koncepti}
\label{subsec:koncepti}


\section{Okruženja (framework) i njihove karakteristike}
\label{sec:okruzenja}
 

\section{Instalacija i pokretanje}
\label{sec:instalacija}

Postoji više načina da se instalira Erlang sa neophodnim paketima.
U ovom odeljku će biti predstavljena instalacija korišćenjem prekompajliranih binarnih fajlova 
za neke operativne sisteme zasnovane na Linuksovom kernelu i pokretanje na jednom od njih, kao 
i instalacija za Windows.

\subsection{Linux}
\label{subsec:instalacijaLinux}

Na operativnim sistemima zasnovanim na {\em Ubuntu}, Erlang se može instalirati sa:
{\em sudo apt-get install erlang}. \\
Nakon uspešne instalacije, Erlang kod je moguće kompajlovati
ili interpretirati i pokretati u interpretatoru.
Interpretator se pokreće kucanjem komande {\em erl} u terminalu, a iz istog
se izlazi sa {\em Ctrl+G} iza kog sledi {\em q} \cite{book_joe}.
Erlang interpretator ima u sebi ugradjen editor teksta koji je baziran na {\em emacs-u} \cite{book_fred}. \\
Kod iz datoteke se kompajluje komandom {\em erlc} i navođenjem imena fajla sa ekstenzijom {\em erl}.
Nakon toga se dobija izvršna datoteka sa ekstenzijom {\em beam} koja se može
pokrenuti uz navođenje adekvatnih flegova. 

\subsection{Windows}
\label{subsec:instalacijaWindows}


\section{Primeri kodova sa objašnjenjima}
\label{sec:primeri}


\section{Specifičnosti}
\label{sec:specificnosti}


\section{Zaključak}
\label{sec:zakljucak}



\addcontentsline{toc}{section}{Literatura}
\appendix
\bibliography{seminarski} 
\bibliographystyle{plain}

\appendix
\section{Dodatak}



\end{document}
